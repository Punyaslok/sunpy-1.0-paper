\section{Introduction}
\label{sec:intro}

Solar physicists rely heavily on software to analyze data. This is especially true today, as solar data sets grow in size and complexity.
The goal of the \sunpyproj, officially founded in March of 2014, is to facilitate and promote the use and development of community-led, free, and open-source\footnote{\url{https://opensource.org/osd}} general purpose tools to maximize the scientific return from solar data. 
To achieve this goal, the project develops and maintains a core Python\footnote{\url{https://www.python.org/}} package (\sunpypkg), supports an ecosystem of affiliated packages (see \autoref{sec:affil_package}) consistent with best practices \citep{Wilson:2014cka}, and engages with the community through mailing lists, chat rooms, tutorials, summer programs, and mentorship. The \sunpyproj shares these goals with the Astropy Project\footnote{\url{https://www.astropy.org}}, which develops the \astropypkg core package \citep{astropy2018} for the astrophysics community.

In recent years, the astronomy and astrophysics communities have increasingly relied on \python, a high-level interpreted  programming language, to collaboratively develop analysis tools for the community.
This choice was motivated by several different factors.
Firstly, the Python programming language is freely-available and open-source, meaning users are not bound by restrictive and costly proprietary licenses.
Additionally, Python is one of the most widely-used programming languages by professional software developers\footnote{According to the 2019 Stack Overflow developer survey(\url{https://insights.stackoverflow.com/survey/2019}), Python is the fourth most popular language among professional developers.} and is also now used by most universities to teach computer science \citep{guo2014}.
As such, new members of the solar physics community are less likely to need to learn a new programming language to become productive researchers and can benefit from the wealth of \python documentation and tutorials available online.

The primary motivating factor for choosing \python is the rich and mature ecosystem of packages for performing scientific analysis and computation. 
The scientific \python ecosystem is supported by foundational packages for manipulation of tabular \citep[\pandas,][]{pandas} and multi-dimensional array\citep[\numpy,][]{numpy} data, general purpose scientific computing \citep[\scipy,][]{scipy}, publication-quality 2D plotting \citep[\matplotlib,][]{matplotlib}. 
These core packages form the backbone of hundreds of additional scientific \python packages, such as \astropypkg, scikit-learn for machine learning and data mining \citep{pedregosa11}, and Dask for parallel and distributed computing \citep{rocklin15}.

Interoperability between all these packages enables interdisciplinary analysis across solar physics, space physics, and astrophysics as well as the greater scientific community.
Notably, packages from the scientific Python ecosystem played a key role in the first detection of a gravitational wave \citep{ligo_scientific_collaboration_and_virgo_collaboration_observation_2016} as well as the recent imaging of a supermassive black hole \citep{collaboration_first_2019}.
The solar physics community stands to benefit immensely by utilizing the \python scientific stack to solve increasingly difficult scientific challenges and produce world-class science.

This paper describes the first stable release (Version 1.0) of the (\sunpypkg) core package.
A previous paper describes v0.5 \citep{Community:2015cy}.
This article is not meant to replace the \sunpypkg documentation but provides an overview of the organization and highlights important functionality.
The full text of the paper, including all of the code to produce the figures, is available in a \github repository\footnote{\url{https://github.com/sunpy/sunpy-1.0-paper}}.