\section{Introduction}
\label{sec:intro}

Research astrophysicists rely on software to analyze increasingly large and complex data sets .
Scientists that study our nearest star, the Sun, are no different.
Solar physicists rely primarily on remote sensing data from both space- and ground-based observatories to measure the properties of our star and deduce the physical mechanisms at work.
In the past, most solar data analysis was performed using Fortran, a general-purpose, compiled programming language designed for scientific and engineering applications.
In the 1980s, the solar physics community made the transition from Fortran to the Interactive Data Language (IDL), a commercial and closed-source programming language.
The community extracted several benefits from this transition.
IDL is a vector-oriented interpreted programming language which enables much faster and therefore cheaper development compared to Fortran.
It also comes packaged with a large number of powerful libraries that support scientific data visualization and analysis.
For similar reasons, the astronomy community also made the transition to IDL which led to the release of the IDL Astronomy User's Library for the solar community to leverage.
Since then, significant and powerful functionality has been developed for solar data analysis as part of SolarSoftWare, a set of freely-available integrated IDL software libraries that provide a common data analysis environment \citep{Freeland:1998we}.

Officially founded in March of 2014, the goal of the \sunpyproj is to provide the core functionality needed for solar data analysis in \python\footnote{\url{https://www.python.org/}}, a high-level interpreted  programming language, and facilitate a new transition from IDL to bring significant and new benefits to the solar community.
The \sunpyproj develops and maintains a community-led, free, and open-source\footnote{\url{https://opensource.org/osd}} core \python package (\sunpypkg), supports an ecosystem of affiliated packages (see \autoref{sec:affil_package}) consistent with best practices \citep{Wilson:2014cka}, and engages with the community through mailing lists, chat rooms, tutorials, summer programs, and mentorship.
The \sunpyproj shares these goals with the Astropy Project\footnote{\url{https://www.astropy.org}}, which develops the \astropypkg core package \citep{astropy2018} for the astrophysics community.

The choice of \python to collaboratively develop analysis tools for the community was motivated by several different factors.
Firstly, the Python programming language is freely-available and open-source, meaning users are not bound by restrictive and costly proprietary licenses.
Additionally, Python is one of the most widely-used programming languages by professional software developers\footnote{According to the 2019 Stack Overflow developer survey(\url{https://insights.stackoverflow.com/survey/2019}), Python is the fourth most popular language among professional developers.} and is also now used by most universities to teach computer science \citep{guo2014}.
As such, new members of the solar physics community are less likely to need to learn a new programming language to become productive researchers and can benefit from the wealth of \python documentation and tutorials available online.

The primary motivating factor for choosing \python is the rich and mature ecosystem of packages for performing scientific analysis and computation. 
The scientific \python ecosystem is supported by foundational packages for manipulating tabular \citep[\pandas,][]{pandas} and multi-dimensional array\citep[\numpy,][]{numpy} data, general purpose scientific computing \citep[\scipy,][]{scipy}, and publication-quality 2D plotting \citep[\matplotlib,][]{matplotlib}. 
These core packages form the backbone of hundreds of additional scientific \python packages, such as \astropypkg for functionality and tools specific to astronomy, scikit-learn for machine learning and data mining \citep{pedregosa11}, and Dask for parallel and distributed computing \citep{rocklin15}. Interoperability between all these packages enables interdisciplinary analysis across solar physics, space physics, and astrophysics as well as the greater scientific community.
Notably, packages from the scientific Python ecosystem played a key role in the first detection of a gravitational wave \citep{ligo_scientific_collaboration_and_virgo_collaboration_observation_2016} as well as the recent imaging of a supermassive black hole \citep{collaboration_first_2019}.
Because the solar community is relatively small\footnote{For reference, out of the $\sim$9,500 members of the American Astronomy Society, approximately 500 are members of the society's Solar Physics Division.} in relation to other scientific communities, it stands to benefit immensely by utilizing the \python scientific stack to solve increasingly difficult scientific challenges and produce world-class science.

This paper describes the first stable release (version 1.0) of the (\sunpypkg) core package.
A previous paper describes v0.5 \citep{Community:2015cy}.
This article is not meant to replace the \sunpypkg documentation but provides an overview of the organization and highlights important functionality.
The full text of the paper, including all of the code to produce the figures, is available in a \github repository\footnote{\url{https://github.com/sunpy/sunpy-1.0-paper}}.