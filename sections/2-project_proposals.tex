\section{Project Organization and Enhancement Proposals}
\label{sec:proj_org}

\begin{deluxetable}{cl}
    \tablecaption{Common Abbreviations and Definitions}
    \tablehead{
        \colhead{Abbreviation} &
        \colhead{Definition}
    }
    \startdata
    AIA & Atmospheric Imaging Assembly \\
    DRMS & Data Record Management System \\
    FITS & Flexible Image Transport System \\
    GOES & \textit{Geostationary Operational Environmental Satellite} \\
    HEK & Heliophysics Event Knowledgebase \\
    HAE & Heliocentric Aries Ecliptic \\
    HCC & Heliocentric Cartesian \\
    HCRS & Heliocentric Celestial Reference System \\
    HEEQ & Heliocentric Earth Equatorial \\
    HGC & Heliographic Carrington \\
    HGS & Heliographic Stonyhurst \\
    HPC & Helioprojective Cartesian \\
    IDL & Interactive Data Language \\
    IRIS & \textit{Interface Region Imaging Spectrograph} \\
    JSOC & Joint Science Operations Center \\
    LTS & Long Term Support \\
    MDI & Michelson Doppler Imager \\
    NASEM & National Academies of Sciences, Engineering, and Medicine \\
    NOAA & National Oceanic and Atmospheric Administration \\
    PEP & Python Enhancement Proposal \\
    SDO & \textit{Solar Dynamics Observatory} \\
    SOHO & \textit{Solar and Heliospheric Observatory} \\
    SEP & SunPy Enhancement Proposal \\
    SPoCA & Spatial Possibilistic Clustering Algorithm \\
    TAI & International Atomic Time \\
    UTC & Coordinated Universal Time \\
    VSO & Virtual Solar Observatory \\
    WCS & World Coordinate System \\
    XRS & X-Ray Sensor \\
    \enddata
\end{deluxetable}

The organization of the \sunpyproj is modeled on the structure of a board-only not-for-profit corporate entity.
It consists of a self-selected board of up to 10 members.
An executive director, elected by the board, leads the core development team and the development of the \sunpypkg  package and supports the development of affiliated packages.
As such, the executive director is also referred to as the lead developer.
A deputy lead developer, a release manager, a group of subpackage maintainers, as well as other volunteers from the developer community, support the lead developer.
Board members serve two-year terms while the lead developer serves one-year terms. No positions have term limits.

The \sunpyproj is formally defined through SunPy Enhancement Proposals (SEPs), which are modeled after the Python Enhancement Proposal (PEP) process\footnote{\url{https://www.python.org/dev/peps/}}.
All SEPs are version-controlled, citable, and publicly available\footnote{\url{https://github.com/sunpy/sunpy-SEP}}.
The first SEP \citep[SEP-0001,][]{sep-0001} defines the scope of an SEP \citep[similar to][]{ape-0001} while the second \citep[SEP-0002,][]{sep-0002} defines the  \sunpyproj organization.

There are three types of SEPs:
\begin{itemize}
    \item \textbf{Standard}: Introduces and describes a new feature, or changes to an existing feature to \sunpypkg, and is meant to function as a high-level technical design document.
    \item \textbf{Process}: Describes a new process in the organization, or changes to an existing process.
    \item \textbf{Informational}: Provides information but does not introduce any new features, changes, or processes.
\end{itemize}

As of the time of writing, there are a total of 9 SEPs which have been approved by the board.
Some notable SEPs have led to the formal adoption of physical units \citep[SEP-0003,][see \autoref{sec:units}]{sep-0003} and a specific scientific time class \citep[SEP-0008,][see \autoref{sec:units}]{sep-0008} throughout the code base.
SEPs have also been used to formally define the affiliated-package program \citep[SEP-0004,][see \autoref{sec:affil_package}]{sep-0004} as well as the release schedule and versioning system \citep[SEP-0009,][see \autoref{sec:release}]{sep-0009}.
