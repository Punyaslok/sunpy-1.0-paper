\section{Project Organization \& Enhancement Proposals}
\label{sec:proj_org}

The organization of the \sunpyproj is modeled on the structure of a board-only not-for-profit corporate entity.
It consists of an up-to 10 member self-selected board.
An executive director, elected by the board, leads the core development team and the development of the \sunpypkg  package and supports the development of affiliated packages
As such, the executive director is also referred to as the lead developer.
A deputy lead developer and release manager, as well as other volunteers from the developer community, support the lead developer.
Board members serve two year terms while the lead developer serves one year terms. Both positions have no term limits.

The \sunpyproj is formally defined through SunPy Enhancement Proposals (SEPs), which are modeled after the Python Enhancement Proposal process\footnote{\url{https://www.python.org/dev/peps/}}.
All SEPs are version-controlled, citable, and publicly available\footnote{\url{https://github.com/sunpy/sunpy-SEP}}.
The first SEP \citep[SEP-0001][]{sep-0001} defines the scope of an SEP \citep[similar to][]{ape-0001} while the second \citep[SEP-0002][]{sep-0002} defines the  \sunpyproj organization.

There are generally three types of SEPs:
\begin{itemize}
    \item \textbf{Standard}: Introduces and describes a new feature, or changes to an existing feature to \sunpypkg, and is meant to function as a high-level technical design document.
    \item \textbf{Process}: Describes a new process, or changes to an existing process, in the organization.
    \item \textbf{Informational}: Provides information and does not introduce any new features, changes, or processes.
\end{itemize}

As of the time of writing, there are a total of 9 SEPs which has been approved by the board.
Some notable SEPs have led to the formal adoption of physical units \citep[SEP-0003][see Section~\ref{sec:units}]{sep-0003}  and a high precision scientific time format \citep[SEP-0008][see Section~\ref{sec:units}]{sep-0008} throughout the code base.
They have also been used to formally define the affiliated package program \citep[SEP-0004][see Section~\ref{sec:affil_package}]{sep-0004}) and the release schedule and versioning system \citep[SEP-0009][see Section~\ref{sec:release}]{sep-0009}).