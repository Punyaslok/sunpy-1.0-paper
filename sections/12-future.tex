\section{Conclusion}
\label{sec:conclusion}

Development of the \sunpypkg core package has been ongoing for 8 years with the adoption of a formal project structure 5 years ago.
The core package has grown to provide significant functionality for a growing number of users.
Significant additional features are currently missing and either under active development on are planned for future development.
These include support for spectra (one dimensional or multidimensional) and spectral fitting, support for multi-dimensional datasets (e.g. slit spectrographs), and providing a standardized approach to metadata.

The project formalization process which defined a board structure has succeeded in providing stability as well as better recognition in the community.
Through the board, significant decisions were able to be made with some level of community consensus such as adopting an official code of conduct\footnote{\url{http://docs.sunpy.org/en/stable/coc.html}} the purpose of which is to ensure that the SunPy community is positive, inclusive, successful, and growing.
This is achieved by expecting that community members be open, considerate of respectful in all interactions.

There are a number of obstacles to the continued growth and success of the project. The inability of the project to identify any significant and long-term funding stream has already been discussed in Section~\ref{sec:support}.
In addition to that, as shown in Figure~\ref{fig:metafig}, the current team of core developer team is relatively small which suggests a lack of robustness to the loss of key developers.
A significant obstacle to the growth of the core developer team is the difficulty in providing the appropriate tools and guidance to users to convert them into active contributors.
Significant additional skills are required of core developers including knowledge of version control, refactoring code for public use, writing user documentation, and unit testing which are not currently prevalent in the solar community.
The \sunpyproj is considering a number of ways to address this issue including providing webinars, internship opportunities and improving online documentation.

The \sunpyproj is now a member of the \python in Heliophysics Community (PyHC)\footnote{\url{heliopython.org}}, whose members contribute to a collection of over fifty \python packages that span every sub-discipline within heliophysics which includes solar physics.
\citet{snakes} provides an overview of the current state of \python packages in Heliophysics as well as calls for a common framework to be developed.
The development of \sunpypkg is consistent with the standards established by the community \citep{pyhcStandards} and many members of the \sunpyproj are explicit signatories.
The goal of this group is to coordinate \python development for Heliophysics to improve interoperability and efficiency which aligns with the goals of the \sunpyproj.
In addition, some of the functionality provided by \sunpypkg subpackages may be developed to be generally useful for all of the sub-disciplines of Heliophysics.
