\section{Conclusion}
\label{sec:conclusion}

Development of the \sunpypkg core package has been ongoing for 8 years with the adoption of a formal project structure 5 years ago.
The core package has grown to provide significant and now mature functionality for a growing number of users.
This milestone release is in recognition of this, combined with a commitment to stability in future releases.
Significant additional features are currently missing and either under active development or are planned for future development.
These include support for generic spectra (one dimensional or multidimensional), multi-dimensional datasets (e.g. slit spectrographs), and a standardized approach to metadata. The roadmap is maintained in a repository\footnote{\url{https://github.com/sunpy/roadmap}} to enable community discussion and additions.

The project formalization process which defined a board structure has succeeded in providing project stability as well as better recognition in the community.
Through the board, significant decisions were able to be made such as joining NumFOCUS and adopting an official code of conduct\footnote{\url{http://docs.sunpy.org/en/stable/coc.html}}, the purpose of which is to ensure that the SunPy community is positive, inclusive, successful, and growing.

There are a number of obstacles to the continued growth and success of the project. The inability of the project to identify any significant and long-term funding stream has already been discussed in \autoref{sec:support}.
In addition to that, as shown in \autoref{fig:metafig}, the current team of core developers is relatively small, which translates to an unhealthy dependence on key developers.
A significant obstacle to the growth of the core developer team is the difficulty in providing the appropriate tools and guidance to users to convert them into active contributors.
Significant additional skills are required of core developers including knowledge of version control, refactoring code for public use, writing user documentation, and unit testing which are not currently prevalent in the solar community.
The \sunpyproj is considering a number of ways to address these issues including providing webinars, internship opportunities, and improving online documentation.

The \sunpyproj is now a member of the Python in Heliophysics Community (PyHC)\footnote{\url{heliopython.org}}, whose members contribute to a collection of over fifty \python packages that span every sub-discipline within heliophysics.
\citet{snakes} provides an overview of the current state of \python packages in heliophysics and calls for a common framework to be developed.
The development of \sunpypkg is consistent with the standards established by the community \citep{pyhcStandards} and many members of the \sunpyproj are explicit signatories to these standards.
The goal of this group is to coordinate \python development for heliophysics in order to improve interoperability and efficiency.
In addition, some of the functionality provided by \sunpypkg subpackages may be developed to be generally useful for all of the heliophysics sub-disciplines.
