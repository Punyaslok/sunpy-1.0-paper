\section{Conclusion}
\label{sec:conclusion}

Development of the \sunpypkg core package has been ongoing for 8 years, with the adoption of a formal project structure 5 years ago.
The core package has grown to provide significant and now mature functionality for a growing number of users.
The release of version 1.0 is a significant milestone, and comes with a commitment to stability in future releases.
Significant additional features are either being actively developed or are planned for future development.
These planned udpates include support for generic spectra (one dimensional or multidimensional), multi-dimensional datasets (e.g., slit spectrographs), and a standardized approach to metadata. The roadmap is maintained in a repository\footnote{\url{https://github.com/sunpy/roadmap}} to enable community discussion and suggestions.

The project formalization process, which defined a board structure, has succeeded in providing project stability as well as better recognition in the community.
Through the board, significant decisions were able to be made such as joining NumFOCUS and adopting an official code of conduct\footnote{\url{http://docs.sunpy.org/en/stable/coc.html}}, the purpose of which is to ensure that the SunPy community is positive, inclusive, successful, and growing.

There are a number of obstacles to the continued growth and success of the project.
The project does not yet have any significant and long-term funding stream.
In addition to that challenge, the current team of core developers is relatively small, which translates to an unhealthy dependence on key developers.
A significant obstacle to the growth of the core developer team is the low familiarity of modern coding best practices in the solar community.
Core developers must be knowledgeable of version control, code refactoring for public use, good user documentation, and unit testing.
It can take considerable effort to provide new contributors with the appropriate guidance and training.
The \sunpyproj is considering a number of ways to address this issue, including providing webinars, internship opportunities, and improving online documentation.

The \sunpyproj is now a member of the Python in Heliophysics Community\footnote{\url{heliopython.org}}, whose members contribute to a collection of over fifty \python packages that span every sub-discipline within heliophysics.
\citet{snakes} provides an overview of the current state of \python packages in heliophysics and calls for a common framework to be developed.
The development of \sunpypkg is consistent with the standards established by the community \citep{pyhcStandards} and many members of the \sunpyproj are signatories to these standards.
The goal of this group is to coordinate \python development for heliophysics in order to improve interoperability and efficiency.
In addition, some of the functionality provided by \sunpypkg subpackages may be broadened to be generally useful for all of the heliophysics sub-disciplines.
