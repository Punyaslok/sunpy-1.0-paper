 \section{Data Types}
\label{sec:data_types}

The \sunpypkg package provides two core data types designed to provide a general, standard, and consistent interface for loading and representing solar data across instruments and missions.
The core data types currently provided are \Timeseries and \Map, which support 1D temporal data and 2D image data, respectively. 
They also include functionality for data visualization and data manipulation. 
This section provides an overview of the \Timeseries and \Map data types.

\subsection{\Timeseries}
\label{sec:timeseries}
Many types of solar data deal with time series. 
For example, the X-ray Sensor aboard the Geostationary Operational Environmental Satellite (GOES) measures the solar integrated X-Ray flux, in two broadband channels, as a function of time. 

Users can interact with and manipulate time series data through the \Timeseries object, which provides a single, consistent interface to manipulate time series data from a variety of instruments. 
Users can add, truncate, re-sample, and combine data within a single \Timeseries or combine multiple \Timeseries together. 
Similar to \Map (Section \ref{sec:map}), \Timeseries has its own visualization methods to allow for easy inspection of the data.

\Timeseries currently supports the data sources listed in Table \ref{tab:instruments} in addition to indices from the National Oceanic and Atmospheric (NOAA) Space Weather Prediction Center (SWPC) that track the solar cycle and its predicted progression. Due to its flexible data structure, it is easy to add additional instruments and data sources to the \Timeseries object.

%%%%%%%%%%%%%%%% TABLE %%%%%%%%%%%%%%%%%%%
\begin{table}
\begin{center}
\begin{tabular}{|p{10cm}|c|c|}
\hline
\textbf{Instrument} & \textbf{Supported by \Timeseries} & \textbf{Supported by \Map} \\
\hline
\hline
\textit{COronal Solar Magnetism Observatory (COSMO)} K-coronagraph (K-Cor) \citep{dewijn12} & & x \\
\hline
\textit{Fermi} Gamma-ray Burst Monitor (GBM) \citep{meegan2009fermi}& x & \\
\hline
\textit{Geostationary Operational Environmental Satellite (GOES)} X-ray Sensor (XRS) \citep{garcia94, hanser96} & x & \\
\hline
\textit{Hinode} X-Ray Telescope (XRT) \citep{golub2008x} & & x \\
\hline
\textit{Interface Region Imaging Spectrograph (IRIS)} Slit Jaw Imager (SJI) \citep{DePontieu2014} & & x \\
\hline
\textit{Nobeyama Radioheliograph (NoRH)} \citep{nakajima1994nobeyama}& x & \\
\hline
\textit{PRoject for Onboard Autonomy (PROBA2)} Large Yield Radiometer (LYRA) \citep{dominique2013lyra}& x & \\
\hline
\textit{PRoject for Onboard Autonomy (PROBA2)} Sun Watcher using Active Pixel System detector and Image Processing (SWAP) \citep{seaton2013swap}&  & x \\
\hline
\textit{Reuven Ramaty High Energy Solar Spectroscopic Imager (RHESSI)} \citep{lin2003reuven}& x & x \\
\hline
\textit{Solar and Heliospheric Observatory (SOHO)} Extreme ultraviolet Imaging Telescope (EIT) \citep{delaboudiniere1995eit}& & x \\
\hline
\textit{Solar and Heliospheric Observatory (SOHO)} Large Angle Spectroscopic COronagraph (LASCO) \citep{brueckner1995large}& & x \\
\hline
\textit{Solar and Heliospheric Observatory (SOHO)} Michelson Doppler Imager (MDI) \citep{scherrer1995solar}& & x \\
\hline
\textit{Solar Dynamics Observatory (SDO)} Atmospheric Imaging Assembly (AIA) \citep{lemen2011atmospheric} & & x \\
\hline
\textit{Solar Dynamics Observatory (SDO)} Helioseismic and Magnetic Imager (HMI) \citep{schou12} & & x \\
\hline
\textit{Solar Dynamics Observatory (SDO)} EUV Variability Experiment (EVE) \citep{woods2010extreme} & x & \\
\hline
\textit{Solar TErrestrial RElations Observatory (STEREO)} Extreme Ultraviolet Imager (EUVI), COronagraph 1 and 2 (COR1/2) for both \textit{STEREO} A and B \citep{howard2008sun} & & x \\
\hline
\textit{Transition Region and Coronal Explorer (TRACE)} \citep{handy99} & & x \\
\hline
\textit{Yohkoh} Soft X-ray Telescope (SXT) \citep{tsuneta1991soft} & & x \\
\hline
\end{tabular}
\end{center}
\caption{The following table outlines the instruments supported by the \Timeseries and \Map objects described in Section \ref{sec:data_types}.}
\label{tab:instruments}
\end{table}
%%%%%%%%%%%%%%%% TABLE %%%%%%%%%%%%%%%%%%%

\subsection{\Map}
\label{sec:map}
Many types of solar data deal with images. For example, the Helioseismic and Magnetic Imager (HMI) instrument aboard the Solar Dynamics Observatory (SDO) maps the magnetic field at the solar photosphere every 45 seconds. 

The \Map class provides a single, consistent interface to analyze 2D image data and relevant metadata from a variety of instruments. 
Users can create a \Map object by providing input data files located locally or fetched via the \sunpypkg data search and retrieval interface called \Fido (see Section \ref{sec:fido}). The \Map object will automatically detect the instrument and subsequently search for the appropriate FITS keywords to infer the coordinate system \citep{refId0, 2006A&A...449..791T}.

The \Map object permits users to plot not only a single image but also overlay multiple images; such functionality is useful in displaying data from instruments with overlapping fields of view. The \Map object also loads source-specific color tables and appropriate image scaling for each instrument.
Users can also combine multiple maps together in a time-ordered sequence. The data in each \Map need not have the same size (number of pixels in each direction) or view the same area of the sky (field-of-view). 

\Map currently supports the data sources listed in Table \ref{tab:instruments} as well as the Helioviewer JPEG2000 image files of for each of these data sources. Due to its flexible data structure, it is easy to add additional instruments and data sources to the \Map object.
