
\subsection{Data Search and Retrieval}
\label{sec:fido}

One of the most important tasks that occurs before any analysis can take place is to search for and retrieve data.
A particular science goal may require data from multiple data providers, each of which may have different methods for data search and retrieval.
This heterogeneity increases the effort required by scientists to get the data they need.
In order to address this issue, the \package{sunpy.net} subpackage provides interfaces to many commonly used data providers and catalogues in solar physics.

\subsubsection{The \Fido Interface}
\label{sec:fido}

The most mature and powerful component of \package{sunpy.net} is the \Fido interface for data search and retrieval.
\Fido provides a unified interface that simplifies and homogenizes search and retrieval by allowing data to be queried and downloaded from multiple data sources simultaneously, irrespective of the underlying client.
Currently, \Fido supports the Virtual Solar Observatory \citep[VSO,][]{2009EM&P..104..315H}, the Joint Science Operations Center (JSOC, see \autoref{sec:drms}), and a number of individual data providers that make their data available via web-accessible resources such as HTTP websites (RHESSI, SDO-EVE, NOAA GOES soft X-ray flux, PROBA2-LYRA, and NOAA sunspot number prediction) and FTP servers (NOAA sunspot number, Nobeyama Radioheliograph).

A \Fido search accesses multiple instruments and all available data providers in a single query.
Search queries optionally include a variety of attributes, such as instrument, time range, and wavelength.
The attributes can be joined using Boolean operators to enable complex queries.
The result of a \Fido query can be inspected and edited before retrieval and is then downloaded via asynchronous and parallel download streams.
\Fido also recognizes failed data downloads and allows for re-requesting files that were not retrieved.

\subsubsection{HEK Client}
\label{sec:hek}

In addition to data download, access to event catalogues are also an important aspect of solar physics research.
One of the largest catalogs is the Heliophysics Event Knowledgebase \citep[HEK,][]{hek}, which provides a searchable database of manually and automatically detected solar features and events (e.g. sunspots, flares, coronal mass ejections). The \sunpypkg package provides an HEK search client that is highly flexible allowing multiple event types and their properties to be queried simultaneously.
For example, it is possible to search for active regions identified by SPoCA (Spatial Possibilistic Clustering Algorithm) \citep{2014AA...561A..29V} above a user-specified size within a given time range.

\subsubsection{Helioviewer Client}
\label{sec:helioviewer}

Finally, \package{sunpy.net} has a Helioviewer\footnote{\url{https://helioviewer.org/}} client which permits the user to query the Helioviewer JPEG2000 image archive.
This client accesses the data archive used by the powerful on-line solar image browsing tool provided by Helioviewer.
It can query and download images as well as request constructed images of solar data from multiple sources.
Planning is underway to migrate this functionality to an affiliated package.
